\section{Resumo do relatório final do TCC 1}
A disciplina Trabalho de Conclusão de Curso I foi realizada no Departamento de Estruturas (SET), no segundo semestre de 2017, portanto o tema escolhido era outro, assim como o professor orientador. Alguns imprevistos impossibilitaram a continuação do TCC 1, em seu percurso natural, no primeiro semestre de 2018. A decisão tomada foi a procura de um novo tema, para ser concluído na duração deste semestre.

\section{Progresso}
Até então foi realizado um profundo estudo sobre os assuntos referentes ao tema, contando com a análise das leis que regem o destino dos resíduos da construção civil, como a Resolução CONAMA 307/2002, a tese de Eder Carlos Guedes dos Santos sobre a avaliação experimental de muros executados com RCD-R, entre diversos outros. Foi possível verificar o interesse da comunidade acadêmica brasileira em aprimorar e incentivar o uso de Resíduos de Construção Civil visto a quantidade de estudos disponíveis.

Os materiais disponíveis sobre geossintéticos também são muitos, em especial o Manual Brasileiro de Geossintéticos e o excelente site \textit{www.geoacademy.com.br}.

\section{Dificuldades encontradas}
A principal dificuldade encontrada foi o prazo de realizar todo o projeto em apenas um semestre, mas com certeza não é algo que inviabiliza a execução do mesmo.