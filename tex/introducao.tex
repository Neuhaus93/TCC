As estruturas de contenção são elementos indispensáveis de uma grande variedade de obras e projetos de engenharia, podendo ser necessárias à construção de pontes, rodovias, ferrovias, prédios em geral, usinas, barragens entre outros. A utilização de solo reforçado em estruturas de contenção é uma alternativa muito utilizada, devido a vantagens como: métodos simplificados de cálculo, fácil adaptação a vários tipo de talude e condições de solo e não exigirem mão de obra especializada e equipamentos caros.
Contudo, \sigla{ESR}{estruturas de solo reforçado} são limitadas pelas especificações de projeto quanto ao tipo de solo a ser utilizado, o que torna inviável em alguns locais a execução deste tipo de obra pela falta de material específico próximo ao local de construção. 

O uso de \sigla{RCD-R}{resíduos de construção e demolição reciclado} como alternativa na utilização de aterro em obras de solo reforçado é uma solução para a limitação citada. O incentivo ao uso de RCD-R em obras de engenharia apresenta benefícios também âmbito do desenvolvimento sustentável, pois oferece um destino consciente para os resíduos gerados pela \sigla{ICC}{Indústria da Construção Civil}.

